\documentclass[11pt]{article}
\usepackage[letterpaper,margin=1in]{geometry}
\usepackage{amsmath,amsthm,amssymb,latexsym}
\usepackage{alltt,enumerate,bbm,eucal,graphicx,color,mathpazo,stmaryrd,bussproofs,mathpartir}
\usepackage{tikz,pgfmath}
\usepgflibrary{shapes}
\usetikzlibrary{arrows,automata,backgrounds}

\usepackage[style=alphabetic]{biblatex}
\addbibresource{bibtex.bib}

\begin{document}

%% Macros %%
\newcommand\one{\textsf 1}
\newcommand\zero{\textsf 0}
\newcommand\cost{\mathcal C}
\newcommand\Lstar{$L^*$}

%%\newtheorem{remark}{Remark}
%%\newtheorem{question}{Q}
\theoremstyle{remark}
\newtheorem{alg}{Algorithm}

\newtheorem{theorem}{Theorem}
\newtheorem{prop}{Proposition}

\setlength\parindent{0in}
\addtolength\parskip{1ex}
\setlength\fboxrule{.5mm}\setlength{\fboxsep}{1.2mm}
\newlength\courseheader
\setlength\courseheader\textwidth
\addtolength\courseheader{-4mm}
\parindent=0pt
\parskip=1ex

\begin{center}
\framebox{\parbox\courseheader{\large
CS6820 Algorithms\hfill December 15, 2021\\
Final Project \hfill G\"oktu\u{g} Saatcioglu (gs724) \hfill Mark Moeller (mdm367)}}
\end{center}
\medskip

These notes explore the problem of sampling random spanning trees on graphs. The
focus will be to present and analyze Wilson's famous algorithm for this problem,
but we take some detours along the way.

\section{Introduction: Spanning Trees}

In this section we introduce notation and explore basics of distributions on
spanning trees.

\subsection{Definitions}

For a connected undirected graph $G = (V, E)$, a collection of edges $T \subseteq E$
form a \emph{spanning tree} if $(V, T)$ is connected and acyclic.

Suppose the edges have nonnegative costs given by a function $w\colon E \to
\mathbb{R^+}$, then the weight of a spanning tree is:
\[ w(T) = \sum_{e\in T} w(e) \]

Prim's and Kruskal's algorithms are classic fast algorithms for finding minimum
weight (i.e., cost) spanning trees. Kozen gives an enlightening general
framework that includes both algorithms \cite{kozen}.

In some situations in distributed systems or networking, however, we may not
want the minimum weight tree necessarily, but instead we want to sample from a
distribution of spanning trees.

% Some stuff about how many trees there are?


\section{Naive First Attempt}

\section{Loop-erased Random Walks}


\section{Wilson's Algorithm}

% Wilson pseudocode
\subsection{Spanning tree with specified root node}

\subsection{Spanning tree with unspecified root}

\section{Analysis of Wilson's Algorithm}



\printbibliography
\end{document}
