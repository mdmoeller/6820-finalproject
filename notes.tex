\documentclass[11pt]{article}
\usepackage[letterpaper,margin=1in]{geometry}
\usepackage{amsmath,amsthm,amssymb,latexsym}
\usepackage{alltt,enumerate}
\usepackage{tikz,pgfmath,algorithm,algorithmic}
\usepgflibrary{shapes}
\usetikzlibrary{arrows,automata,backgrounds}

\usepackage[style=numeric]{biblatex}
\addbibresource{bibtex.bib}

\begin{document}

%% Macros %%
\newcommand\one{\textsf 1}
\newcommand\zero{\textsf 0}
\newcommand\cost{\mathcal C}

%%\newtheorem{remark}{Remark}
%%\newtheorem{question}{Q}
\theoremstyle{remark}
\newtheorem{alg}{Algorithm}

\newtheorem{theorem}{Theorem}
\newtheorem{defn}{Definition}
\newtheorem{prop}{Proposition}

\setlength\parindent{0in}
\addtolength\parskip{1ex}
\setlength\fboxrule{.5mm}\setlength{\fboxsep}{1.2mm}
\newlength\courseheader
\setlength\courseheader\textwidth
\addtolength\courseheader{-4mm}
\parindent=0pt
\parskip=1ex

\begin{center}
\framebox{\parbox\courseheader{\large
CS6820 Algorithms\hfill December 15, 2021\\
Final Project \hfill G\"oktu\u{g} Saatcioglu (gs724) \hfill Mark Moeller (mdm367)}}
\end{center}
\medskip

These notes explore the problem of sampling random spanning trees on graphs. The
focus will be to present and analyze Wilson's famous algorithm for this problem,
but we take some detours along the way.

We have also implemented the algorithms presented and run them on various
graphs. Details of our implementation are presented in Section~\ref{imp}.
As a 6820 final project, our approach is a hybrid of the ``Extra topic'' and ``Coding
project'' project types.

\section{Introduction: Spanning Trees}

In this section we introduce notation and explore basics of distributions on
spanning trees.

\subsection{Definitions}

\begin{defn}
For a connected undirected graph $G = (V, E)$, a collection of edges $T \subseteq E$
form a \emph{spanning tree} if $(V, T)$ is connected and acyclic.
\end{defn}

Suppose the edges have nonnegative costs given by a function $w\colon E \to
\mathbb{R^+}$, then the weight of a spanning tree is:
\[ w(T) = \prod_{e\in T} w(e) \]

\begin{defn}
A spanning tree for $G$ is called the \emph{minimum spanning tree} if it has the
minimum weight of all spanning trees of $G$.
\end{defn}

Prim's and Kruskal's algorithms are classic fast algorithms for finding minimum
spanning trees. Kozen gives an enlightening general framework that includes both
algorithms \cite{kozen}.

In some situations in distributed systems or networking, however, we may not
want the minimum spanning tree necessarily, but instead we want to sample
randomly from a distribution of spanning trees. We will present an algorithm due
to Wilson \cite{wilson} for doing this in Section~\ref{wilson}. Then we will see
how it can be extended to consider graphs with weights. In this case, the
probability of sampling a given tree will be proportional to its weight.

% Some stuff about how many trees there are?


\section{Naive First Attempt}\label{naive-attempt}

We might be tempted to try to sample random spanning trees by modifying
Kruskal's algorithm. That is:
\begin{itemize}
\item Choose a random order on edges, perhaps based on edge weights
\item Choose edges from the list in order, skipping an edge if it would make a
cycle
\end{itemize}

Unfortunately, we will not achieve the appropriate distributions following this
path. For a counterexample, consider the triangle graph $G = (\{a,b,c\},
\{(a,b),(b,c),(a,c)\})$, with edge weights:
\[w(e) = \begin{cases}
        2 & \text{ for } e = (a,b)\\
        1 & \text{ for } e = (b,c)\\
        1 & \text{ for } e = (a,c)
        \end{cases}\]

We observe that for this graph there are 3 spanning trees:
$T_0 = \{(a,b), (a,c)\}$,
$T_1 = \{(a,b), (b,c)\}$,
$T_2 = \{(a,c), (b,c)\}$, with weights:

\[w(T) = \begin{cases}
        2 & \text{ for } T = T_0\\
        2 & \text{ for } T = T_1\\
        1 & \text{ for } T = T_2
        \end{cases}\]

Therefore we want to sample $T_0$ or $T_1$ each with probability 2/5, and $T_2$
with probability 1/5. But if we sample the trees by the method described above
(i.e., choosing edges propropotional to their weight---and for this graph just
pick the first two), then we get the following distribution:

\begin{align*}
(a,b), (a,c)\text{ with Pr }= 1/2 \cdot 1/2 = 1/4, (T_0 \text{ is selected})\\
(a,c), (a,b)\text{ with Pr }= 1/4 \cdot 2/3 = 1/6, (T_0 \text{ is selected})\\
(a,b), (b,c)\text{ with Pr }= 1/2 \cdot 1/2 = 1/4, (T_1 \text{ is selected})\\
(b,c), (a,b)\text{ with Pr }= 1/4 \cdot 2/3 = 1/6, (T_1 \text{ is selected})\\
(b,c), (a,c)\text{ with Pr }= 1/4 \cdot 1/3 = 1/12, (T_2 \text{ is selected})\\
(a,c), (b,c)\text{ with Pr }= 1/4 \cdot 1/3 = 1/12, (T_2 \text{ is selected})
\end{align*}

So we get:
\begin{align*}
\text{Pr}(T_0) = 1/4 + 1/6 = 10/24\\
\text{Pr}(T_1) = 1/4 + 1/6 = 10/24\\
\text{Pr}(T_2) = 1/12 + 1/12 = 1/6
\end{align*}

which is the wrong distribution (see above). We will see in the next section that
\emph{loop-erased random walks} are the key to sampling spanning trees.

\section{Loop-erased Random Walks}

The definitions in this section allow use of directed or undirected graphs. We
assume a stochastic transition matrix. If the edge weights of our graph do not
already form a stochastic matrix, we can obtain one by normalizing the weights
and (possibly) adding self-transitions for slack in probabilities. These
self-transitions will have no affect on the ensuing algorithms, for reasons that
will be clear shortly.

\begin{defn}
A \emph{random walk} on a graph $G=(V,E)$ is a sequence of vertices $v_i \in V$ of a Markov
chain $M$ whose state is a node $v \in V$ and which, at each step, transitions
to a neighbor of $v$ with probabilities given by $M$.
\end{defn}

A \emph{loop-erased random walk} (or \emph{self-avoiding random walk}) is
obtained by deleting the cycles of a random walk. That is, if in the course of
random walk, we find ourselves at a vertex for a second time the sequence is
deleted to the first visit of the node and we proceed as if it were that first
visit. Evidently, a loop-erased random walk must be finite since any repeating
nodes are deleted.

Some random walk algorithms measure their running time in
comparison to the \emph{cover time} of the graph, which is the expected number
of steps for a random walk (not loop-erased!) to reach all of the vertices the
first time.

% TODO hitting time

\section{Wilson's Algorithm}\label{wilson}
\subsection{Spanning tree with specified root node}
\begin{algorithm}
\caption{Wilson's algorithm for given root}
\label{alg:root}
\textbf{Input: }Graph $G=(V,E)$, Root $r \in V$ \\
\textbf{Output: }Spanning tree $T$ \\
\begin{algorithmic}[1]
\STATE T = \{\}                   // Set of nodes which are in the tree
\STATE next = \{\}                // Map from nodes to their successor
\FOR{ each v in V}
\STATE $u \leftarrow v$
\WHILE{u not in T}\label{walk}
\STATE next[u] $\leftarrow$ samplesuccessor(u)
\STATE u $\leftarrow$ next[u]
\ENDWHILE \label{endwalk}
\STATE $u \leftarrow v$
\WHILE{u not in T} \label{adjoin}
\STATE T.add(u)
\STATE $u \leftarrow $next[u]
\ENDWHILE \label{endadjoin}
\ENDFOR
\STATE \textbf{return} next
\end{algorithmic}
\end{algorithm}

Wilson's algorithm for sampling a spanning tree with a specified root is given
in Algorithm~\ref{alg:root}. It works as follows:
\begin{itemize}
\item Initialize a tree to contain only the root.
\item For each node, perform a random walk until we hit the tree, and add that
path into the tree.
\item Return the resulting tree.
\end{itemize}


We note that the loop on lines~\ref{walk}-\ref{endwalk} implements the
loop-erased random walk in a subtle way.  Specifically, the erasure of a cycle is
\emph{not} a special case, in the sense that if we end up back where we started,
the successor we write down will overwrite the one that was previously there
anyway. This will orphan the rest of the links in the cycle (in the ``next''
array), but this is okay because these links cannot be read by the walk (which only
reads from ``next'' after writing to it) or the tree-adjoining section
(lines~\ref{adjoin}-\ref{endadjoin}) without first being rewritten by the
walk.

\subsection{Spanning tree with unspecified root}
\begin{algorithm}
\caption{Wilson's algorithm with unspecified root}
\label{alg:noroot}
\textbf{Input: }Graph $G=(V,E)$, probability $\epsilon$ \\
\textbf{Output: }Spanning tree $T$ \\
\begin{algorithmic}
\STATE T = \{\}                   // Set of nodes which are in the tree
\STATE next = \{\}                // Map from nodes to their successor
\STATE num\_roots $\leftarrow$ 0
\FOR{$v$ in $V$}
\STATE $u \leftarrow i$
\WHILE{u not in T:}
\IF{chance($\epsilon$)}
\STATE next[u] $\leftarrow$ null
\STATE T.add(u)
\STATE num\_roots += 1
\IF{num\_roots = 2}
\STATE \textbf{return} Fail
\ENDIF
\ELSE
\STATE next[u] $\leftarrow$ samplesucc(u)
\STATE $u \leftarrow $next[u]
\ENDIF
\STATE $u \leftarrow v$
\ENDWHILE
\WHILE{u not in T} \label{adjoin}
\STATE T.add(u)
\STATE $u \leftarrow $next[u]
\ENDWHILE
\ENDFOR
\STATE \textbf{return} next
\end{algorithmic}
\end{algorithm}

Wilson's algorithm for sampling a spanning tree without a specified root is
shown in Algorithm~\ref{alg:noroot}.

\section{Analysis of Wilson's Algorithm}

% Talk about markov chain proof

\section{Implementation}\label{imp}

We discuss our implementation of Wilson's Algorithm~\ref{alg:root} and Wilson's Algorithm~\ref{alg:noroot} along with an implementation of Kruskal's minimum spanning tree algorithm where the order on the edges being processed by the algorithm can be chosen according to three different strategies. These strategies are as follows. Firstly, we can order all edges by increasing edge cost to get the classic Kruskal's algorithm for minimum spanning trees. Secondly, we can output a random permutation of the edges to get a simple ``randomized'' version of Kruskal's algorithm. Thirdly, we can output a random ordering of edges where the probability that we pick a certain edge weight in the ordering is proportional to its edge weight and the edge weight of all unpicked edges so far. The last two options correspond to a naive approach for sampling a random spanning tree as outlined in Section~\ref{naive-attempt}.

Our implementation is written OCaml \texttt{4.11.0} and can be located in the \texttt{code} folder of the public GitHub repository for this project. We rely on the Dune build system and Jane Street's Core and Core\_Bench libraries. \iffalse TO-DO: REFERENCES \fi Details on installing OCaml, the relevant dependencies and compiling the program can be found in the \texttt{README}. At a high-level, we have four modules.

\begin{itemize}
\item \texttt{Parser}: The parser takes as an input a file name that describes the undirected graph we would like to analyze and outputs a representation of that graph that we use for our algorithms. The input file describes a graph as two integers $n$ and $m$ which correspond to the number of vertices and edges of the graph respectively. Furthermore, the user enters $m$ lines in the format $i,j,p$ which is interpreted as an edge $(i,j)$ with weight $p$. The parser output a set of vertices $\{0,\dots,n-1\}$ and a adjacency list representation of the graph.
\item \texttt{Random\_tree}: This module implements Wilson's Algorithm~\ref{alg:root} and Wilson's Algorithm~\ref{alg:noroot} by taking as input a graph from the \texttt{Parser} module and running the corresponding algorithms as described in Section~\label{wilson} and Wilson's original paper \cite{wilson}. The output is a set of edges which can then be written to a file where each edge is written to a line of the file.
\item \texttt{Kruskal}: The \texttt{Kruskal} module implements the three Kruskal algorithms described in the beginning paragraph of this section. Each algorithm takes as input a graph from the \texttt{Parser} module and sorts the edges accordingly. Then, Kruskal's minimum spanning tree algorithm is run and we output a set of edges that corresponds to a minimum spanning tree with respect to the order we picked. This module also has a function to generate all spanning trees of a given graph. The implementation is simple and takes exponential time, we generate all permutation of the edge set of a graph and for each permutation run Kruskal's minimum spanning tree algorithm. To remove isomorphic spanning trees we canonicalize each spanning tree by sorting the edge set using the lexicographic ordering on $\mathbb{N}\times\mathbb{N}$.
\item \texttt{Tester}: The \texttt{Tester} module runs a set of tests to measure the empirical distribution of the spanning trees found by each algorithm and the average running time of each algorithm. To achieve this the module exposes a single function called \texttt{test} that accepts as input a graph from the \texttt{Parser} module and an integer \texttt{iter} which is the number of times a given test should be run. Then, the test suite will generate all spanning trees using \texttt{Kruskal} and run each algorithm from \texttt{Random\_tree} and \texttt{Kruskal} \texttt{iter} many times. The results of the frequencies of each spanning tree is saved to a file while the running times of each algorithm is printed to the screen. This is actually a quirk of the Core\_Bench as it has limited functionality and we could find no other micro-benchmarking library for OCaml. (Here, micro-benchmarks are necessary as none of our tests on average run for more than a couple of hundred micro-seconds.)
\end{itemize}

The generated program, \texttt{tree\_sampler}, can be used to run either any one of the algorithms individually or the test suite. The test suite is used to run our experimental evaluation which we describe next.

\section{Evaluation}\label{eval}



\printbibliography
\end{document}
